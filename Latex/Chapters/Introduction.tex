% Chapter Template

\chapter{Introduction} % Main chapter title

\label{Introduction} % Change X to a consecutive number; for referencing this chapter elsewhere, use \ref{ChapterX}

%----------------------------------------------------------------------------------------
%	SECTION 1
%----------------------------------------------------------------------------------------

While global warming is an increasingly worrying subject, solutions appear. One of them would allow the construction sector to become less polluting. This solution is based on the use of wood in construction. The wood is used for many applications. First as a combustible and then for tools fabrication, construction, or paper manufacturing. It is an interesting material, on all fronts. It is sustainable (1\si{\cubic\meter} catch 1 ton of CO2) and findable on every area. It allows a large panel of construction element. Even if it must be used after some treatments, the material cost, is one of the cheapest. However, wood is subject to weathering, and depending on weather conditions, will not have the same properties. The present document aims to study this material, depending on the temperature, humidity or the loads applied on it. Indeed, a wood beam submitted at snow loads, hydric changes, or seasonal gap of temperature must resist whatever the modification from climatic conditions. This report focuses on wood fracture mechanics under these conditions.
\newline 
Thanks to this document, many essentials' information are regrouped, to understand how the fracture mechanic can be studied. Basic equation and manual resolution are presented, but with the numerous illustrations, this document is accessible for neophyte person interested by the subject.
For now, the study of the wood mechanical fracture will be done in opening mode and the experiments are carried out using Digital Image Analysis systems, in particular Python program, in order to compute MatchID images recorded. The subject is too important to be treated in only six months, this is the reason of Mode I focus and absence of comparing tools. 
\newline
MMCG test, such as the DCB and CTS ones, are also presented, and they are carried out on specimens from different species placed in different climatic conditions. The beam which composes your frame can come from several areas. That is why, the chosen species allow to have a panel of results. Indeed, tropical wood, temperate other, with different densities, are studied and their mechanical resistances permit to have the big picture of wood behavior. 
\newline
Tests are done in Portugal at NOVA Scool of Science and Technology, Universidade Nova de Lisboa. One interest is also to compared this results with previous ones and with digital modelization software tools, as Abaqus in this current case. An expected result is the negative impact of temperature decrease and moisture content increase on wood behavior.