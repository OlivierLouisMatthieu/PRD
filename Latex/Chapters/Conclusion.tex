% Chapter Template

\chapter{Conclusion} % Main chapter title

\label{Conclusion} % Change X to a consecutive number; for referencing this chapter elsewhere, use \ref{ChapterX}

%----------------------------------------------------------------------------------------
%	SECTION 1
%----------------------------------------------------------------------------------------

In conclusion of this work, by analyzing the behavior of Padouk and Okoume species submitted to different moisture content, this study has shown how this climatic effect can affect wood fracture mechanics.
\newline
The tests done at different moisture contents, climatic room, 20\% and 30\%, for several MMCG geometry specimens, coupled to the use of DIC with MatchID software and python processing were determinant to obtained results. Even if the final experiments do not present tests at changing temperature or the specimen collapsed by fatigue loads, the focus was put on moisture content parameter, allowing to have more samples to treat, in order to have a better idea of a potential trend. The choice of the focus on this parameter more than others, was linked to the little research done on this parameter while numerous are leaded on temperature variation and fatigue impact. Whereas, the Silver Fir was not able to be treated, so the experiments do not compare tropical species with temperate ones. But a comparison was done on previous work. 
The link between European species and Okoume in terms of density show a difference of behavior. Okoume looks like a less affected wood regarding moisture content impact on its mechanical characteristics. At the contrary, the final trend visible on the Padouck available results, was not compared with european species, due to the absence of fracture mechanic tests at different moisture content for this denser wood. So the results shown and the interpretation can be discussed, but according to this work, Padouck seems to have better characteristics at 15\% moisture content, before a decrease of the energy release rate with the 20\% moisture content treeshold overrun. 
\newline
The impossibility to do real comparison due to the shapes of the samples, the species and the moisture content variation, intrinsic to this project, does not allow a final answer. But the uses of this work are numerous. Some issues were found and solutions are proposed to avoid future problems on the test. By changing one parameter as the moisture content or the species, it will be possible to reuse the developed Python code and many determined parameters, in order to find a pattern linking moisture content to fracture mechanics in mode 1. An obvious conclusion is that wood behavior is submitted to too many evolution to found an empiric model, and every specimen tested will be a particular case. This is the reason why much experiments must be done on this exceptional material.