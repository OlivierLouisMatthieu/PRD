Plan 

Abstract i
1 Introduction 1
2 Litterature review
2.1 Wood behavior 
2.1.1 Wood Structure and composition 
2.1.2 An anisotropic and orthotropic material 
2.2 Wood species
2.2.1 Tropical species
2.2.2 Temperate woods
2.3 Wood experimentations specimens : 
2.3.1 DCB (Double Cantilever Beam) test
2.3.2 CTS
2.3.4 MMCG	 
2.3.5 Climatical effects on the test 
2.4 Rupture mechanics on wood
2.4.1 Elaboration areas
2.4.2 Mechanical field
2.4.3 Coefficient of intensity  
2.4.4 Energetic methods
2.4.5 Method of the complacency
2.5 Softwares and analysing tools 
2.5.1 Image Correlation knowledges
2.5.2 Analysis Software
2.5.3 Programation tools
3. Pre and Post processing work preparation
3.1 Abaqus software
3.1.1 Geometry creation
3.1.2 Boundary condition
3.1.3 Zone of Interest 
3.1.4 Mesh and link to finite element analysis
3.2 Python program
3.2.1 Initialization and input values
3.2.2 Alpha parameter
3.2.3 Determination of the CTOD
3.2.4 Crack length analysis
3.2.5 Compliance and Energy values
3.3 Grips
3.3.1 Arcan model
3.3.2 Mode I chosen grips
%Katia explanation
3.4 Specimen preparation
3.4.1 Notch and Precrack
3.4.2 Moisture Content determination
3.4.3 Chosen experiments
%only at different MC 
4. Experimental work and Result
4.1 Experimental Setup
4.1.1 Camera and MatchID Setup
4.1.2 Hydraulic Press and grips Setup
4.1.3 MMCG specimen last preparation and Moisture Content reach
%painted specimen and names
4.1.4 Difficulties and issues
4.2 Results
4.2.1 Settings to obtain datas
%MatchID setting to process and Python modification
4.2.2 Results at ambient temperature and ambient humidity
4.2.3 Results with specimen at a moisture content around 30% 
4.2.4 Results with specimen at an intermediar moisture content
4.3 Analysis of the Results
4.3.1 Comparison between species
4.3.2 Comparison between moisture content levels
4.3.3 Differences with previous works
4.3.4 Way of amelioration and next step
5 Conclusion


