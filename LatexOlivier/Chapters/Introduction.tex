% Chapter Template

\chapter{Introduction} % Main chapter title

\label{Introduction} % Change X to a consecutive number; for referencing this chapter elsewhere, use \ref{ChapterX}

%----------------------------------------------------------------------------------------
%	SECTION 1
%----------------------------------------------------------------------------------------

Wood is a natural resource present on all continents. In France, it represents 15 million hectares and 3.3 million hectares in Portugal. It is a natural, renewable, low energy cost, recyclable and biologically degradable material. Today, global warming is a major concern and the use of wood in the construction sector would allow to become less polluting. It is used for many applications. It can be used as energy, in timber frames and constructions, for packaging, in furniture and joinery, in the paper industry or in the chemical industry for example. In addition, the cost of wood is one of the lowest. One of the defects of wood is its natural origin. This gives it a great source of variability in its characteristics. It is subject to weathering and, depending on the climatic conditions, it will not have the same properties. This is why it must often be used after a certain treatment.

Thanks to this document, a lot of essential information is gathered, in order to understand how fracture mechanics can be studied. The study of the mechanical fracture of wood will be done in mode I and in mixed mode on silver fir specimens of type 2MCG. The proposed work consists mainly in conducting measurements carried out by kinematic field measurements obtained by the method known as image correlation or DIC. The use of DIC method with MatchID software, combined with subsequent processing using Python, will play a crucial role in obtaining the results. Two Python programs will be compared in order to verify the acuity of each method. This project consists in modeling the observed behavior, both from the point of view of crack initiation and propagation. The subject is too important to be treated in only six months, that is why there will be only one specie of wood and variations in wood moisture content will not be a study parameter taken into account. The tests are performed in Portugal at the NOVA School of Science and Technology, Universidade Nova de Lisboa. The interest is also to compare these results with previous ones and with numerical modeling software, such as Abaqus.

This manuscript is structured in 4 chapters: 
The first chapter is a literature review study on the wood material and on the mechanics of the rupture. It is a question of reminding the elements which can influence the behavior of wood with regard to the rupture, in particular on the constituent elements of the wood material on the macroscopic and microscopic level. The basics of fracture mechanics and the DIC method are presented for wood material.
The second chapter explains all the preparations made beforehand in order to carry out and analyze the tests performed.
In the third chapter, various tests in mode 1 are performed on 2MCG specimens. The measurements performed will allow to obtain the different lengths and openings of cracks and the energy restitution rates.
The last chapter will present the cracking in mixed mode for 3 angles. It is to examine the propagation of cracking, always with the same image correlation technique, with the MMCG specimen which has the particularity of having an adapted geometry that allows to control the propagation of cracks in wood.

