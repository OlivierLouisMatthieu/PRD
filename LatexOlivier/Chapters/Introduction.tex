% Chapter Template

\chapter{Introduction} % Main chapter title

\label{Introduction} % Change X to a consecutive number; for referencing this chapter elsewhere, use \ref{ChapterX}

%----------------------------------------------------------------------------------------
%	SECTION 1
%----------------------------------------------------------------------------------------

Wood is a natural resource present on all continents. In France, it represents 15 million hectares and 3.3 million hectares in Portugal. It is a natural, renewable, low energy cost, recyclable and biologically degradable material. Today, global warming is a significant concern and using wood in the construction sector would make it less polluting. It is used for many applications. It can be used as energy, in timber frames and constructions, for packaging, furniture and joinery, in the paper industry or the chemical industry, for example. In addition, the cost of wood is one of the lowest. One of the defects of wood is its natural origin. This gives it a great source of variability in its characteristics. It is subject to weathering and, depending on the climatic conditions, it will not have the same properties. This is why it must often be used after a specific treatment.

Thanks to this document, much essential information is gathered to understand how fracture mechanics can be studied. The mechanical fracture of wood will be studied in mode I and in a range of mixed-mode I+II on silver fir specimens of type 2MCG. The proposed work consists mainly in conducting measurements carried out by kinematic field measurements obtained by the method known as image correlation or DIC. Integrating the DIC method through the MatchID software and subsequent Python-based data processing will be instrumental in attaining the desired outcomes. Two Python programs will be compared in order to verify the acuity of each method. This project consists of modelling the observed behaviour, both from the crack initiation and propagation point of view. The study will focus solely on a single wood species, and variations in wood moisture content will not be considered. The tests were carried out in a universal testing machine with a homemade Arcan fixture. This data can be used in future studies to develop computational simulations of fracture testing to determine fracture parameters from inverse methods.

This manuscript is divided into four chapters. The first chapter is a literature review on wood morphology and fracture mechanics. It aims to provide an overview of the factors influencing wood behaviour concerning its constituent elements on both the macroscopic and microscopic levels. The fundamentals of fracture mechanics and the DIC method are also introduced within the context of wood material.
The second chapter is dedicated to the methodology and preparation required for conducting and analysing the preliminary tests. It explains the procedures and steps involved in detail.
Moving on to the third chapter, a series of mode I tests are conducted on 2MCG specimens. These tests involve measurements to determine crack lengths, openings, and energy restitution rates. The obtained data will contribute to the overall understanding of crack propagation in this mode.
The fourth chapter focuses on crack propagation in mixed mode for three different angles (15$^\circ$, 30$^\circ$, and 45$^\circ$). The study utilises the same image correlation technique to examine crack propagation patterns in the MMCG specimen, which features a geometry specifically designed to control crack propagation in wood.
Overall, the manuscript comprehensively explores wood material behaviour, fracture mechanics, testing methodologies, and crack propagation characteristics in both mode I and mixed mode I+II conditions.

