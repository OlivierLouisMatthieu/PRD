\chapter{Experimental work and result in mixed mode}
\label{Chapter2}

\section{Introduction}

In this section, the results obtained in mixed mode are presented. The decoupling of the modes has allowed to obtain the share of mode 1 and 2. The imposed displacement complacency method of equation x has been used to calculate the different energy restitution rates. The deformation maps, force vs crack opening curves, energy restitution rate vs crack length for different cases are presented below. The values of the GI/GII ratios and the GI - GII difference are also given. The calculation of the difference is intended to show the behaviour and evolution of the two modes with respect to the variation of alpha.

\section{Deformation maps}

Figure x shows the deformation maps for the three tilt angles of the specimen. The crack propagation path, although not perfectly straight, does not appear to change direction. As in Chapter 3, cracks tend to propagate according to the orientation and inclination of the fibres.

\section{Force - Crack opening curve}

For $\alpha=15$ degrees the crack openings along the x-axis are negligible compared to those along the y-axis. Along x, these values are just slightly higher than the values obtained with $\alpha=0$.

The gap between the crack openings along the x-axis and the y-axis is considerably reduced for a$\alpha=45$.

\section{Critical energy restitution rate}

\section{Comparisons and analysis}

Table x shows the different values of G. The differences in mode 1 and 2 are compared.
Figure x shows the difference in the values of the energy restitution rates GI - GII according to the different mixing angles. We can clearly see the decrease of this difference between the mode 1 and mode 2 share with the increase of the mixing angle.

After analysis of the tests for the different mix ratios, the results obtained gave the following information:

For all the mix angles studied, the mode 1 share is always higher than the mode 2 share

The energy restitution rate values from mode 1 decrease with increasing degree of mixing angle, while the energy restitution rate values from mode 2 increase

The GI/GII ratio as a function of crack length has a constant evolution as shown in Figure x. This may indicate that the values characterising the cracking obtained are intrinsic to the species studied

The difference between the GI-GII values decreases with increasing degree of the mixing angle (Figure x) justifying at the same time the complete decoupling of the mixed modes of failure. In figure x, the results of the GI-GII values obtained for the tests with an angle of 15, 30 and 45 degrees are shown. We can see a decreasing evolution of this difference. This can be explained by the fact that when we approach an angle of 45 degrees, the values of the crack openings along x and y tend to get closer. In mixed mode, the crack opening is no longer one-dimensional since the specimen is stressed at an angle that induces the cohabitation of two modes, which leads to a projection along both axes of the crack opening. Obtaining the two components Ux and Uy led to the calculation of GI and GII . In other words, this allows the decoupling of the two modes. The difference between the values of the two modes is therefore maximal when the mixing angle is 15 degrees . It will decrease until the angle of 45 degrees where it is minimal. Between 0 degrees and 45 degrees mode 1 is superior to mode 2. It is probably after a mixing angle of 45 degrees that mode 2 is supposed to be superior to mode 1.