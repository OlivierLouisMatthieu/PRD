% Chapter Template

\chapter{Conclusion} % Main chapter title

\label{Conclusion} % Change X to a consecutive number; for referencing this chapter elsewhere, use \ref{ChapterX}

%----------------------------------------------------------------------------------------
%	SECTION 1
%----------------------------------------------------------------------------------------

In conclusion, this study aimed to investigate the fracture parameters of softwood species, specifically silver fir. Wooden specimens were used and mounted on a tension-compression testing machine, using the specially designed steel Arcan system.. The research involved a comprehensive examination of both microscopic and macroscopic constituents of wood, as well as the application of linear fracture mechanics formulas. A thorough literature review was conducted, focusing on mechanical specimens used in temperate species.

The utilization of digital image correlation (DIC) method with MatchID software, coupled with subsequent processing using Python, played a crucial role in obtaining the results. The experiments were primarily conducted in open mode (mode I) on MMCG specimens. The critical energy restitution rate for each specimen was calculated using the compliance method. The average $G_{max}$ values were compared to those of other temperate species, revealing significant similarities.

Furthermore, the study investigated mixed mode fracture for various mixing angles, following a similar methodology as the open mode tests. By decoupling modes I and II, the energy restitution rate calculations indicated a predominance of mode I over mode II. Additionally, it was observed that the energy restitution rate for mode I decreased with increasing mixing angle, while the energy restitution rate for mode II increased.

Looking ahead, the geometry of the MMCG specimen, the Arcan system, and the Python code developed can be readily applied to future tests. It would be particularly interesting to conduct tests in both mode I and mixed mode using different wood species, as well as varying levels of humidity and temperature. This would contribute to a more comprehensive understanding of wood behavior. However, it is worth noting that due to the intricate nature of wood's behavior, empirical models may not be feasible, and each tested sample will likely present unique characteristics. Therefore, numerous experiments are necessary to explore the complexities of this exceptional material.