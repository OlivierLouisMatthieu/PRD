% Chapter Template

\chapter{Conclusion} % Main chapter title

\label{Conclusion} % Change X to a consecutive number; for referencing this chapter elsewhere, use \ref{ChapterX}

%----------------------------------------------------------------------------------------
%	SECTION 1
%----------------------------------------------------------------------------------------

In conclusion, this study aimed to investigate the fracture parameters of silver fir, a type of softwood species. The Arcan fixture was utilised to apply both mode I and mixed-mode I+II fracture loading. Wooden specimens were specifically manufactured along the RL propagation planes. Digital image correlation (DIC) with MatchID software was employed to analyse the deformation patterns, and a Python code was developed for subsequent data processing to obtain fracture parameters, including crack tip opening displacement and crack length evaluation during the fracture tests.

The experiments primarily focused on open mode (mode I) tests conducted on MMCG specimens. The critical energy release rate for each specimen was calculated using the compliance method. Comparative analysis of the average $G_{max}$ values with those of other temperate species revealed significant similarities, providing valuable insights into the fracture behaviour of silver fir.
Additionally, the study investigated mixed mode fracture by varying the mixing angles, following a similar methodology as the open mode tests. By decoupling modes I and II, the calculations of the energy release rate indicated a predominance of mode I over mode II. Moreover, it was observed that the energy release rate for mode I decreased with increasing mixing angle, while the energy release rate for mode II increased.

Moving forward, the geometry of the MMCG specimen, the Arcan system, and the developed Python code can be readily applied to future tests. It would be particularly intriguing to conduct tests in both mode I and mixed mode I+II using different wood species whilst also considering variations in humidity and temperature levels. Such comprehensive investigations would enhance our understanding of wood behaviour. However, it is important to acknowledge that due to the intricate nature of wood's behaviour, developing empirical models may be challenging, and each tested sample will likely exhibit unique characteristics. Therefore, numerous experiments are necessary to explore the complexities of this biological material.