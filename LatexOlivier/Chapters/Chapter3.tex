\chapter{Experimental work and result in mode I}
\label{Chapter3}

\section{Introduction}

In this section, we present and analyse the force vs the crack opening curve obtained with the DIC method and python code.  Then the evolution of the crack length curves along the x-axis is obtained as a function of displacement. Finally, the energy restitution rates for the MMCG specimens are calculated and analysed.

\section{Force - crack opening curve}

The crack opening curves are presented in Figures x. It is obtained from the displacement maps and using the methodology presented in method 1.

Add some plots later

\section{Force-crack length curve}

Figure x shows an example of the evolution of the force as a function of the crack length. The crack length increases gradually and continuously before rupture. It is the design of its shape that allows the stable propagation of the crack. Indeed, with this specimen, cracks progress slowly until failure occurs at maximum force.

\section{Critical energy restitution rate}

\section{Discussion}

Table \ref{fig:fig37} compares the different values obtained in this study with literature values obtained on temperate species. These are numerous, especially in mode I. In order to provide an objective comparison, the discussion focuses on temperate species with the same density and an initial crack oriented in the RL direction.


\begin{table} \centering
	\begin{tabular}{ccccccc}
		\toprule % horizontal line at the top of the table
		& References & Wood species & Test type & Orientation & Density & $G_{max}(J/m^2)$\\\midrule
		& \cite{Mambili2018} & Normal poplar & 2MCG & RL & 0.35-0.5 & 1287\\\midrule
		& \cite{Mambili2018} & Tension poplar & 2MCG & RL & 0.35-0.5 & 430\\\midrule
		& \cite{Mambili2018} & Normal white fir & 2MCG & RL & 0.53 & 761\\\midrule
		& \cite{Mambili2018} & Compression white fir & 2MCG & RL  & 0.53 & 1169\\\midrule
		& \cite{Odounga2018phd} & Okoumé & 2MCG & RL & 0.39-0.5 & 317\\\midrule
		& \cite{Odounga2018phd} & Iroko & 2MCG & RL & 0.56-0.7 & 323\\\midrule
		& \cite{Odounga2018phd} & Padouk & 2MCG & RL & 0.7-0.88 & 255\\\midrule
		& \cite{Odounga2018phd} & Iroko & 2MCG & RL & 0.56-0.7 & 323\\\midrule
		& \cite{Reiterer2002} & Spruce & WS & RL & 0.479 & 323\\\midrule
		& \cite{Reiterer2002} & Alder & WS & RL & 0.553 & 255\\\midrule
		& \cite{Reiterer2002} & Oak & WS & RL & 0.553 & 348\\\midrule
		& \cite{Reiterer2002} & Ash & WS & RL & 0.701 & 551\\
		\bottomrule % horizontal line at the bottom of the table
	\end{tabular}
	\caption{Comparison of mean max G values for specimens in the literature}
	\label{fig:fig37}
\end{table}

\section{Conclusion}