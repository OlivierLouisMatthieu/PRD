\chapter{Experimental work and result in mode I}
\label{Chapter3}

\section{Introduction}

In this chapter, the opening mode failure behaviour is investigated using the DIC method. We present and analyse the force vs the crack opening curve obtained with the DIC method and python code.  Then the evolution of the crack length curves along the x-axis is obtained as a function of displacement. Finally, the energy restitution rates for the MMCG specimens are calculated and analysed with some other articles.

\section{Experimental set-up and method}

The Landmark Servohydraulic compression tensile machine, with a capacity of 100 kN, can be seen in Figure \ref{fig:Setup0°}. The camera and Arcan device are also clearly visible. The part of the specimen with the speckle pattern was illuminated so that the black dots could be correctly distinguished for image correlation. The camera was mounted on a tripod in order to capture stable images during the tests. The movable jaw of the test machine is controlled in fixtures and is equipped with a load sensor.
An x-camera coupled with a 60mm Nikkor lens was used for image acquisition. The camera has a resolution of x (H) × x (V) and a sensor size of x. The front of the lens was positioned at a working distance of 285 mm with an aperture of f/11 and an exposure time of 60 milliseconds. The displacement of the machine was 0.02mm/s and the camera had a frequency of 1 Hz, i.e. 1 frame per second.

\begin{figure}[htp]
	\centering
	\includegraphics[width=12cm]{Setup0°}
	\caption{Experimental set-up}
	\label{fig:Setup0°}
\end{figure}

The kinematic fields acquired by image correlation (e.g., subset size, subset step, ldots) and numerical differentiation (e.g., strain windows size) algorithms can be significantly impacted by the DIC configuration settings. Since they will determine the spatial resolution and accuracy linked to the DIC measurements, both in displacement and strain fields, these settings constitute fundamental parameters.  In order to balance resolution and spatial resolution, a parametric research was done to support the DIC setting for the current application. The MatchID 2D Parametric Module was used to conduct this investigation.

\begin{table}[]
	\centering
	\begin{tabular}{c c}
		\hline
		Correlation   Coefficient: & ZNSSD \\ 
		Interpolation order: & Bicubic Splines \\ 
		Transformation order: & Quadratic \\
		Subset size: & 31 \\
		Step size: & 10 \\
		Strainwindow size: & 5 \\ 
		Strain interpolation: & Q4 \\ 
		Strain Convention: & Green-Lagrange \\ \hline
	\end{tabular}
	\caption{MatchID parameters used}
	\label{tab:MatchID_param}
\end{table}

The range of values defined in this performance study, including the subset size ($f_s$), subset step, affine and quadratic displacement shape functions, the size of the strain windows, and the order of the polynomial fitting function, is defined by the table \ref{tab:MatchID_param}. It is believed that the pre-selected range of values is reflective of the permissible DIC setup parameter range.

\section{Force-displacement curve}

Normally four distinct parts are observed on a force-displacement curve:

\begin{itemize}
	\item A small area visible at the beginning of the curve which corresponds to the setting up of the loaded specimen. 
	\item A practically linear zone, terminated by the critical force FC , due to the elastic part of the loading with a static crack front.
	\item A third part characterised by the observation of a set of critical force peaks FC responsible for a fairly clear initiation of the crack. The increasing peaks validate the zone of stability of the crack.
	\item Finally, a last part where the rupture of the material occurs following the application of a breaking force translating the final instability of the crack.
\end{itemize}

 In the figure \ref{fig:e0e1_Pdelta}, we don't see the setting up of the loaded specimen because a slight preload has been applied to the specimen so that it does not move during the beginning of the test. In addition, it is also quite difficult to see the critical FC force peaks because of the noise caused by the load sensor.

\begin{figure}[htp]
	\centering
	\includegraphics[width=9cm]{e0e1_Pdelta}
	\caption{Characteristic P-Delta curve}
	\label{fig:e0e1_Pdelta}
\end{figure}


\section{Deformation maps}

Typical deformation map ($\epsilon$yy) obtained with the DIC method is shown in Figure \ref{fig:fig38}.
During the tests we noticed that the cracks tend to propagate according to the orientation and inclination of the fibres.
In figure \ref{fig:fig38} for specimen e2e2, we assume a small fibre inclination which could explain the crack orientation that we see.

\begin{figure}[htp]
	\centering
	\includegraphics[width=9cm]{fig38}
	\caption{Example of deformation map specimen e2e2 ($\epsilon$yy)}
	\label{fig:fig38}
\end{figure}

Strain maps are used to best locate the position of the crack tip by graphic reading.
Thus, thanks to the deformation maps we are able to obtain the crack length on different stages. It is then possible to check whether method 1 is working correctly.
The blue points figure \ref{fig:fig39} and \ref{fig:fig40} for tests e2e2 and e4e1 are the points read graphically with the deformation maps. It can be seen that the blue points follow the curve of method 1 correctly, so we can consider that method 1 works and that we can use it for all the tests.

\begin{figure}[htp]
	\begin{minipage}[c]{.46\linewidth}
		\centering
		\includegraphics[width=7cm]{fig39}
		\caption{$\overline{VD}$ and $VD_{th}$ with Joao's data}
		\label{fig:fig39}
	\end{minipage}
	\hfill%
	\begin{minipage}[c]{.46\linewidth}
		\centering
		\includegraphics[width=7cm]{fig40}
		\caption{$\overline{VD}$ and $VD_{th}$ with MMCG data}
		\label{fig:fig40}
	\end{minipage}
\end{figure}


\section{Force - Crack tip opening curve}

The user must select the "CODpair". In order to provide accurate displacement, the chosen pair of subsets must be as close as possible to the crack tip while yet being sufficiently removed to prevent information loss (if they are positioned inside the crack). Databases are updated using the selected CODpair following this study. Due to the selected COD pair being in blue, a plot was produced that displays the wI shapes that were used, as seen on figure \ref{fig:CTOD_example}. In order to demonstrate for each specimen that the COD pair picked could not have been more exact, the plot also compared this curve to the one created using the lower and upper COD pairs.

\begin{figure}[htp]
	\centering
	\includegraphics[width=9cm]{CTOD_example}
	\caption{Crack Tip Opening Displacement}
	\label{fig:CTOD_example}
\end{figure}

All the crack opening curves in mode I (displacement of the two lips of the crack) are shown in figure \ref{fig:COD_modeI}. 

\begin{figure}[htp]
	\centering
	\includegraphics[width=16cm]{COD_modeI}
	\caption{Crack Opening Displacement}
	\label{fig:COD_modeI}
\end{figure}

\section{Crack length-Displacement curve}

For method 1 by compiling the evolution of a(t) as a function of the images recorded for several alpha values (\ref{fig:CTOD_example}), it is possible to get an idea of the alpha value required to obtain the best a(t). Indeed, the alpha parameter must be as small as possible for the evolution of the crack length to be complete. So, for each crack length in the sample, it is possible to eliminate several candidates. In this example \ref{fig:CTOD_example}, it is possible to eliminate the use of curves with a(t)<70mm. These alpha values do not allow the entire crack length to be studied. To distinguish between the black and turquoise curves, you can use the displacement map by placing the position of the crack tip with a red dot on a figure. For different stages, we can then see which curve corresponds best to the position of the crack tip. Here, the black curve was chosen.

To obtain a(t) using method 2, we need to read $a_1$ and $a_f$ graphically and choose a pair of COD lines that are not damaged by nans values.

\begin{figure}[htp]
	\centering
	\includegraphics[width=9cm]{Cracklength_modeI_example}
	\caption{Crack length evolution depending on alpha}
	\label{fig:Cracklength_modeI_example}
\end{figure}

All the crack length curve in function of displacement in mode I are shown in figure x and y. The crack length increases gradually and continuously before rupture. It is the design of its shape that allows the stable propagation of the crack. Indeed, with this specimen, cracks progress slowly until failure occurs at maximum force.

\begin{figure}[htp]
	\centering
	\includegraphics[width=16cm]{crack_method1}
	\caption{Crack length evolution depending on alpha}
	\label{fig:Cracklength_modeI_example}
\end{figure}

\begin{figure}[htp]
	\centering
	\includegraphics[width=16cm]{crack_method2}
	\caption{Crack length evolution depending on alpha}
	\label{fig:Cracklength_modeI_example}
\end{figure}

\section{Critical energy restitution rate}

\section{Discussion}

Table \ref{fig:fig37} compares the different values obtained in this study with literature values obtained on temperate species. These are numerous, especially in mode I. In order to provide an objective comparison, the discussion focuses on temperate species with the same density and an initial crack oriented in the RL direction.


\begin{table} \centering
	\begin{tabular}{ccccccc}
		\toprule % horizontal line at the top of the table
		& References & Wood species & Test type & Orientation & Density & $G_{max}(J/m^2)$\\\midrule
		& \cite{Mambili2018} & Normal poplar & 2MCG & RL & 0.35-0.5 & 1287\\\midrule
		& \cite{Mambili2018} & Tension poplar & 2MCG & RL & 0.35-0.5 & 430\\\midrule
		& \cite{Mambili2018} & Normal white fir & 2MCG & RL & 0.49 & 761\\\midrule
		& \cite{Mambili2018} & Compression white fir & 2MCG & RL  & 0.49 & 1169\\\midrule
		& \cite{Odounga2018phd} & Okoumé & 2MCG & RL & 0.39-0.5 & 317\\\midrule
		& \cite{Odounga2018phd} & Iroko & 2MCG & RL & 0.56-0.7 & 323\\\midrule
		& \cite{Odounga2018phd} & Padouk & 2MCG & RL & 0.7-0.88 & 255\\\midrule
		& \cite{Angellier2017} & Douglas fir & DCB & RL  & 0.54 & 784\\\midrule
		& \cite{Angellier2017} & White fir & DCB & RL  & 0.49 & 570\\\midrule
		& \cite{Xavieretal2014} & Pinus Pinaster & DCB & RL & 0.543 & 270\\\midrule
		& \cite{Reiterer2002} & Spruce & WS & RL & 0.479 & 337\\\midrule
		& \cite{Reiterer2002} & Alder & WS & RL & 0.510 & 244\\\midrule
		& \cite{Reiterer2002} & Oak & WS & RL & 0.553 & 348\\\midrule
		& \cite{Reiterer2002} & Ash & WS & RL & 0.701 & 551\\\midrule
		\bottomrule % horizontal line at the bottom of the table
	\end{tabular}
	\caption{Comparison of mean max G values for specimens in the literature, 2MCG: Mixed Mode Crack Growth, DCB: Double Cantilever Beam, WS: Wedge Splitting test, RL: Radial Longitudinal}
	\label{fig:fig37}
\end{table}

\section{Conclusion}