\chapter{Experimental work and result in mode I}
\label{Chapter3}


\begin{table} \centering
	\begin{tabular}{ccccccc}
		\toprule % horizontal line at the top of the table
		& References & Wood species & Test type & Orientation & Density & $G_{max}(J/m^2)$\\\midrule
		& \cite{Mambili2018} & Normal poplar & 2MCG & RL & 0.35-0.5 & 1287\\\midrule
		& \cite{Mambili2018} & Tension poplar & 2MCG & RL & 0.35-0.5 & 430\\\midrule
		& \cite{Mambili2018} & Normal white fir & 2MCG & RL & 0.53 & 761\\\midrule
		& \cite{Mambili2018} & Compression white fir & 2MCG & RL  & 0.53 & 1169\\\midrule
		& \cite{Odounga2018phd} & Okoumé & 2MCG & RL & 0.39-0.5 & 317\\\midrule
		& \cite{Odounga2018phd} & Iroko & 2MCG & RL & 0.56-0.7 & 323\\\midrule
		& \cite{Odounga2018phd} & Padouk & 2MCG & RL & 0.7-0.88 & 255\\\midrule
		& \cite{Odounga2018phd} & Iroko & 2MCG & RL & 0.56-0.7 & 323\\\midrule
		& \cite{Reiterer2002} & Spruce & WS & RL & 0.479 & 323\\\midrule
		& \cite{Reiterer2002} & Alder & WS & RL & 0.553 & 255\\\midrule
		& \cite{Reiterer2002} & Oak & WS & RL & 0.553 & 348\\\midrule
		& \cite{Reiterer2002} & Ash & WS & RL & 0.701 & 551\\
		\bottomrule % horizontal line at the bottom of the table
	\end{tabular}
	\caption{Comparison of mean max G values for specimens in the literature}
	\label{fig:fig13}
\end{table}
