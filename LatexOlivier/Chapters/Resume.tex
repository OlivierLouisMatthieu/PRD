% Chapter Template

\chapter{Resume} % Main chapter title

\label{Resume} % Change X to a consecutive number; for referencing this chapter elsewhere, use \ref{ChapterX}

%----------------------------------------------------------------------------------------
%	SECTION 1
%----------------------------------------------------------------------------------------
A l’aube d’une transition énergétique sans précédent, de nombreux domaines d’activité se tournent vers des matériaux propres et le bois est le parfait candidat. Le principal avantage de ce matériau est son comportement bénéfique vis-à-vis de l’environnement. Mais c’est aussi son plus gros défaut. C’est pourquoi, à la suite de nombreux travaux, nous continuons à nous intéresser à son comportement face aux variations de température, d’humidité ou de chargement mécanique. La fissuration de différentes espèces de bois en fonction de la variation de ces paramètres, permet de mieux comprendre son comportement pour une utilisation futur accrue. En effet, la finalité de cette étude, serait l’exploitation d’espèces encore peu utilisées, en prouvant que leur comportement est apte à une utilisation en construction. Afin de mener à bien ce projet, la fissuration des bois tropicaux (Iroko, Padouck et Okoumé) est comparée à celle d’une espèce Européenne (Sapin Pectiné). La géométrie des éprouvettes étudiées sont celles développées dans de précédents travaux, elles portent le nom de MMCG (Mixed Mode Crack Growth).
	
Des méthodes modernes d’analyses d’images, permettent une étude précise de la fissuration au cours du temps en fonction de la force appliquée à l’éprouvette soumise à des humidités internes différentes. Les résultats de cette étude proviennent de l’enregistrement de l’évolution de la fissure, capturée et traitée par une de ces méthodes numériques appelée DIC (Digital Image Corrélation). Le taux de restitution d’énergie G déterminé à la suite de la fissuration en mode I, dépend de la longueur de la fissure, de la force appliquée et de l’ouverture de la fissure. La méthode de la Complaisance sera utilisée pour obtenir G. Ces résultats expérimentaux, après traitement par des outils tel que MatchID ou MatLab sont comparés à des résultats théoriques, calculés via le logiciel Abaqus.
	
Enfin, chaque résultat a été mis en parallèle avec d’anciens travaux menés sur le sujet, afin de comparer ces valeurs expérimentales et théoriques. Ainsi, la cohérence des résultats pouvant être discutée, il est désormais assuré que ces derniers, peuvent être le support pour l’utilisation de ces essences dans le domaine de la construction en environnements variable.