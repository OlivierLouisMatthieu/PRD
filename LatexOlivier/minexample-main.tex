\usepackage{graphicx}
\usepackage{makeidx}
\usepackage{datetime2}
\usepackage{alltt}
\usepackage[colorlinks]{hyperref}

\makeindex

\latex{%
  \ifpdf
   \pdfinfo{
       /Author (Nicola Talbot)
       /Title  (Creating a LaTeX Minimal Example)
       /CreationDate (D:20081114000000Z)
       /ModDate (\pdffilemoddate{minexample-main.tex})
       /Subject (LaTeX)
       /Keywords (LaTeX;debug)
    }
  \fi
}

% for latex2html
\providecommand{\hyperpage}[1]{#1}
\providecommand{\meta}[1]{\emph{#1}}
\providecommand{\cls}[1]{\texttt{#1}}
\providecommand{\sty}[1]{\texttt{#1}}
\providecommand{\cs}[1]{\texttt{\textbackslash#1}}
\providecommand{\env}[1]{\texttt{#1}}

\setbaseurl{http://www.dickimaw-books.com/latex/minexample}

\title{Creating a LaTeX Minimal Example}
\author{Nicola L C Talbot}
\date{\today\ (version 1.3)}

\begin{document}
\maketitle

\begin{abstract}
Debugging LaTeX errors often requires creating a minimal (or
minimum) example. This is particularly important when posting a bug
report or request for help, as it facilitates the diagnostic
process. Creating a minimal example will often help you identify the
problem, without having the hassle of posting your query and waiting
until you get a reply. This document illustrates how to create a
minimal example.  See also \htmladdnormallink{Need More
Help?}{http://www.dickimaw-books.com/latex/novices/html/help.html}
and
\htmladdnormallink{Packages and Document Classes}{http://www.dickimaw-books.com/latex/admin/html/packages.shtml}.

The home page for this document is \url{\baseurl/}.
\casemedia{}{}{This document is also available in PDF formatted either in 
\htmladdnormallink{A4 for printing}{\baseurl/minexample-a4.pdf} or 
\htmladdnormallink{6inx4in for on-line viewing}{\baseurl/minexample-screen.pdf}. }
The source code for this document is available as
\htmladdnormallink{a ZIP archive}{\baseurl/minexample.zip}.
\end{abstract}

Copyright \copyright\ 2008 Nicola L. C. Talbot Permission is granted to copy,
distribute and/or modify this document under the terms of the GNU
Free Documentation License, Version 1.2 or any later version
published by the Free Software Foundation; with no Invariant
Sections, no Front-Cover Texts, and no Back-Cover Texts. A copy of
the license is included in the section entitled \htmlref{\qt{GNU Free
Documentation License}}{sec:fdl}. 

\tableofcontents

\setnode{introduction}
\section{Introduction}
\label{sec:intro}

A minimal example\footnote{A minimal example is sometimes called a
\qt{minimal working example} (MWE), but this is a bit confusing as
the purpose of a minimal example is to illustrate unexpected or
non-working behaviour.}\ is the smallest possible complete document that
illustrates a problem. A minimal example file should not include any
packages or code that do not contribute to the problem, but must
include a document class and the document environment. 

There are two approaches to creating a minimal example: \qt{building
up} and \qt{hacking down}. This document illustrates both
approaches. Creating the minimal example may lead you to the
solution, but if you are still stuck, you can then post the minimal
example. (Remembering first to search for the solution in the
\htmlref{documentation}{sec:documentation} and on the Internet, for
example, in newsgroup archives or on forums or Q\&A sites.)

Some package authors (including me) from time-to-time\footnote{How
frequently will often depend on the author's work commitments.}\ read messages on sites such as
\htmladdnormallink{The \LaTeX\
Community}{http://www.latex-community.org/forum/}, 
\htmladdnormallink{\TeX\ on
StackExchange}{http://tex.stackexchange.com/} or on newsgroups such
as \htmladdnormallink{comp.text.tex}{news:comp.text.tex}, so if
you have a problem you can't solve it's generally a good idea to
post your query in one of those places (remembering to paste the
contents of your minimal file in your message). If you've made a
mistake in your code, then someone may be able to point it out,
which may mean that you get a reply quicker than you would if you
posted your query directly to the author. Also, other people will be
able to see your query and learn from it. Remember that no one is
being paid or is otherwise obliged to answer your query, so be
careful not to make your query sound like a demand or an accusation. 

Note that when posting your query, you also need to give a brief
description of the problem, and list the methods that you have tried
to trace the problem. Don't go into a long rambling description of
your project, as it generally doesn't help to identify the problem,
and too much information can put people off reading your request.
It's also a good idea to first search the
\htmladdnormallink{comp.text.tex
archives}{http://groups.google.com/group/comp.text.tex/} or use the
search box on sites like \htmladdnormallink{The \LaTeX\
Community}{http://www.latex-community.org/forum/},
\htmladdnormallink{\TeX\ on
StackExchange}{http://tex.stackexchange.com/} to find out if anyone
else has asked the same question. If you ask a
\htmladdnormallink{frequently asked
question}{http://www.tex.ac.uk/faq}, you may get a curt reply from
people who are tired of answering the same old question, so check
first. 

\setnode{buildingup}
\section{Building Up}
\label{sec:buildingup}

With the building up approach, you start with the document:
\begin{verbatim}
\documentclass{article}
\begin{document}
\end{document}
\end{verbatim}
and add to it until you encounter your problem. If your problem
requires the use of \cs{chapter}, then replace \cls{article} with
either \cls{report} or \cls{book}. 

This section illustrates the building up approach with an example.
Suppose your problem document looks something like:
\begin{verbatim}
\documentclass{myuniversityscustomclass}

\usepackage[french,USenglish]{babel}
\usepackage[mmddyyyy]{datetime}
\usepackage{nonstandardpackage}
\usepackage{anothernonstandardpackage}
% lots of other packages that may or may not be standard

% lots of your own definitions

\author{John Doe}
\title{Astounding Discoveries}

\begin{document}
\maketitle
\tableofcontents
\listoffigures
\listoftables

% 300 or so pages of text, graphics, tables, bibliography and
% sundry other stuff

\end{document}
\end{verbatim}

Let's suppose that your problem is that the date on the title page
looks like November 14, 2008, but you are expecting it to appear in
the form 11/14/2008. You have already checked that you used the
option \texttt{mmddyyyy} when you loaded the \sty{datetime} package,
so what's gone wrong? 

Since you haven't used \cs{date}, the date on the title page is
generated using \cs{today}, so the fault must lie in the definition of
\cs{today}. It looks like it might be a bug in the \sty{datetime} package, so
what should you do? This happens to be one of my
packages,\footnote{although it's now been replaced by \sty{datetime2}} but if
you send me your entire 300 page document plus several hundred
graphics files and a large bibliography file, I won't be best
pleased. Aside from filling up my inbox, I don't have your
university's custom class file, nor am I likely to have the 
non-standard packages installed on my system, so I won't be able to test
the document. At which point you'll either get a request for a
minimal example, or I'll think \qt{forget that, I'll look at it some
other day} (or words to that effect) and then several days, or
possibly weeks, later you'll get a request for a minimal
example.\footnote{Actually, these days I'll just ask you to post
your bug report on \htmladdnormallink{my bug report
form}{http://www.dickimaw-books.com/bug-report.html}.} 

You've already worked out that the problem must lie with the command
\cs{today}. So that needs to go in the minimal example. You want to use
the \sty{datetime} package to change the format of this command, so that
package needs to go in the minimal example, with the package options
you have specified in your original document:
\begin{verbatim}
\documentclass{article}
\usepackage[mmddyyyy]{datetime}
\begin{document}
\today
\end{document}
\end{verbatim}

Call this file, say, \texttt{test.tex}, and run \LaTeX\ on it. Have a look at
the output. The output looks fine, so perhaps one of the other
packages you have loaded has caused the problem. One by one try each
of the packages you have in your problem document, in the same
order. If adding the package has no effect on the output, then
delete that package from the test file, and go on to the next one.
For example, the problem document loads the babel package, so add
that package to the test file using the same options that you used
in your problem document. The minimal example should now look like: 
\begin{verbatim}
\documentclass{article}
\usepackage[french,USenglish]{babel}
\usepackage[mmddyyyy]{datetime}
\begin{document}
\today
\end{document}
\end{verbatim}
Now run it through \LaTeX, and check the result. The output has
changed to November 14, 2008, instead of 11/14/2008. This test file
now reproduces the error, but is only six lines instead of several
hundred or possible thousand lines. 

What next? Check the \sty{datetime}
\htmlref{documentation}{sec:documentation} to see if it mentions
the \sty{babel} package. The \sty{datetime} documentation comes in both PDF and
HTML format. Most PDF and HTML viewers have a function that allows
you to search the document or page for a given word, so search for
the word \qt{babel}. This should lead you to the sentence which states
that the babel package must be loaded before the \sty{datetime} package.
Check the test file. In this test file, the \sty{babel} package has been
loaded first.

Now what?  In this case, there is a FAQ for the \sty{datetime} package
(\url{http://www.dickimaw-books.com/faqs/datetimefaq.html}) so that's the next
place to look. This FAQ covers the most commonly used packages that
I have written.\footnote{or more precisely, it covers the packages
that I get the most post about.} If you look at the table of contents for the
\sty{datetime} section, you should see the entry \qt{The date is in another
language or in the wrong format}. This fits the problem, so click on
that link and have a look at the answer. The answer indicates that
there was a bug in an earlier version of the \sty{datetime} package that
caused a problem when used in conjunction with the \sty{babel} package,
but the bug has been fixed. So the next thing to do is check which
version you are using. Add the command \cs{listfiles} to the test file: 
\begin{verbatim}
\listfiles
\documentclass{article}
\usepackage[french,USenglish]{babel}
\usepackage[mmddyyyy]{datetime}
\begin{document}
\today
\end{document}
\end{verbatim}
At the end of the log file there should now be a list of all the
files that have been loaded, along with their release dates and
versions. Check the version of the \sty{datetime} package. Is it the
latest version? If not, download the latest version and try again.
If it is the latest version, then send the author (me, in the case
of the \sty{datetime} package) the test file and its log file.
If you check the package documentation, you should either find the
author's contact details or a link to a bug reporting tool.

If the conflicting package is one that is not publicly available
(for example, it's your university's custom package that can only be
downloaded from a restricted site) then send your query to the
author of that package. If the conflicting package is publicly
available, but is not on \htmladdnormallink{CTAN}{http://ctan.org/}, 
then specify from where it can be downloaded. 

\setnode{hackingdown}
\section{Hacking Down}
\label{sec:hackingdown}

The \htmlref{previous section}{sec:buildingup} illustrated how to
build up a minimal example. This section shows how to hack down a
minimal example. Again, we are going to start with a 300 page
document which contains many images, tables and a bibliography. 
\begin{verbatim}
\documentclass{myuniversityscustomclass}

\usepackage{nonstandardpackage}
\usepackage{anothernonstandardpackage}
% lots of other packages

\usepackage{glossaries}

% lots of your own command and environment definitions

\newglossaryentry{minex}{name={Minimal Example},
description={A small document illustrating failing behaviour},
text={minimal example}}

% lots more glossary definitions

\author{John Doe}
\title{Astounding Discoveries}

\begin{document}
\maketitle
\tableofcontents
\listoffigures
\listoftables

% 300 or so pages of text, graphics, tables and
% sundry other stuff

% Somewhere in the document is the following:
A \gls{minex is essential when encountering a \TeX\ or \LaTeX\ 
error you don't understand.

% Lots more text, figures, tables and a bibliography
\end{document}
\end{verbatim}
This document is causing the following error:
\begin{verbatim}
Runaway argument?
{minexam is essential when encountering a \TeX \ or \LaTeX \^^Merror
\ETC.
! Paragraph ended before \\@gls was complete.
<to be read again>
                   \par
\end{verbatim}
Suppose you don't understand what the error is or whereabouts in the
document it is occurring\footnote{Actually, in this example it
should print the line number in the error message since \cs{gls} is a
short command, but not all runaway argument errors give a helpful
line number, so let's pretend it hasn't. }.

Since you don't know what command is causing the problem, you can't
use the approach illustrated in the previous section. So you will
need to use the hacking down approach.

Before doing anything else, \textbf{make a copy} of the problem document.
Call the copy, say, \texttt{test.tex}, and only edit this. Don't start
messing around with the original document until you've solved the
problem, otherwise you could lose your work! 

One way of tracking down the problem is to use a binary search.
Suppose your document contains 1000 lines of source code, then go to
line 500 of your test document (i.e.\ half-way through it) and insert
the line\footnote{\LaTeX\ will finish the document when it reaches the
first \cs{end}\texttt{\{document\}}, and ignore everything that comes after it.}:
\begin{verbatim}
\end{document}
\end{verbatim}
(Make sure you don't put it inside a group or environment.)

Now pass the test document to LaTeX. You may get some warning
messages as a result of omitting half the document, but don't worry
about that for now.
\begin{itemize}
 \item If the error still occurs, then the problem is in the first
half of the document. In which case, delete everything after the
first \verb|\end{document}| (in your test file), and repeat the process. 

 \item If the error goes away, then the problem is in the second
half of the document. In which case, delete everything after
\verb|\begin{document}| up to, and including, the first
\verb|\end{document}| (in your test file), and repeat the process. 
\end{itemize}

Continue the process until you only have one paragraph left in your
document. If this has an \cs{input} or \cs{include} command, first
remove (or comment out) the command. If the problem goes away then
the error is in that file, in which case replace the \cs{input} or
\cs{include} command with the contents of the relevant file in your
test file, and repeat the process. Once you have finished, it's a
good idea to add \cs{listfiles}. 

Let's suppose we now have a test file that looks like:
\begin{verbatim}
\listfiles
\documentclass{myuniversityscustomclass}

\usepackage{nonstandardpackage}
\usepackage{anothernonstandardpackage}
% lots of other packages

\usepackage{glossaries}

% lots of your own command and environment definitions

\newglossaryentry{minex}{name={Minimal Example},
description={A small document illustrating failing behaviour},
text={minimal example}}

% lots more glossary definitions

\begin{document}

A \gls{minex is essential when encountering a \TeX\ or \LaTeX\ 
error you don't understand.

\end{document}
\end{verbatim}

It may be that you can now identify the problem, but let's suppose
you still don't know what's wrong. The next thing to do is to remove
unnecessary information in the preamble. If you have defined any
commands or environments in the preamble that aren't used in the
problem paragraph, then delete them. This includes any new theorems
or glossary entries and so on. In this example, the problem
paragraph contains a glossary entry, so keep the definition for that
entry, and delete all the others: 
\begin{verbatim}
\listfiles
\documentclass{myuniversityscustomclass}

\usepackage{nonstandardpackage}
\usepackage{anothernonstandardpackage}
% lots of other packages

\usepackage{glossaries}

\newglossaryentry{minex}{name={Minimal Example},
description={A small document illustrating failing behaviour},
text={minimal example}}

\begin{document}

A \gls{minex is essential when encountering a \TeX\ or \LaTeX\ 
error you don't understand.

\end{document}
\end{verbatim}
Now, one by one, remove any packages that aren't contributing to the
problem. Each time you remove a package, run the test file through
\LaTeX. If the error goes away, then put the package back in. If
removing a package causes an \qt{Undefined control sequence} error,
then remove the undefined command as well. If the problem goes away,
add the command and package back again. For example, if I remove the
line:
\begin{verbatim}
\usepackage{glossaries}
\end{verbatim}
then I will get an error as neither \cs{newglossaryentry} nor
\cs{gls} will be defined. If I remove those commands, the original
error message will go away. So I have to leave those commands in and
keep the glossaries package in the test file. 

Next, try substituting the class file for the \cls{article} or
\cls{report} class file. If the error goes away, then the original
class file is contributing to the problem, in which case put it back
again. If this class file is not publicly available (for example, it
may be an in-house class file, such as a university thesis, which
has restricted access) then contact the author of the class file,
and send the test file and log file. (Remembering, of course, to
first search the \htmlref{documentation}{sec:documentation}.)

If you followed all of the above steps, then the test file should
now look like:
\begin{verbatim}
\listfiles
\documentclass{article}

\usepackage{glossaries}

\newglossaryentry{minex}{name={Minimal Example},
description={A small document illustrating failing behaviour},
text={minimal example}}

\begin{document}

A \gls{minex is essential when encountering a \TeX\ or \LaTeX\ 
error you don't understand.

\end{document}
\end{verbatim}
In this example, you should now be able to work out that there is a
missing closing brace to the argument of \cs{gls}. If, however, you
still can't work out the problem, then (assuming that you've already
read the \htmlref{documentation}{sec:documentation} and searched
relevant forums or newsgroup archives) copy and paste the test file
in a message to somewhere like \htmladdnormallink{\TeX\ on
StackExchange}{http://tex.stackexchange.com/} or
\htmladdnormallink{The \LaTeX\ Community}{http://www.latex-community.org/forum/}
or \htmladdnormallink{comp.text.tex}{news:comp.text.tex}.

\setnode{additionalfiles}
\section{Additional Files}
\label{sec:additionalfiles}

You've tried \htmlref{building up}{sec:buildingup} or
\htmlref{hacking down}{sec:hackingdown} a minimal example, but the
problem is caused by an additional file which you can't copy and
paste into the minimal example file, so what do you do? 

If the file is a graphics file, replace the command with a rule of
the same dimension. For example, if your image is 4in wide by 3in
high, then replace: 
\begin{verbatim}
\includegraphics{myImage}
\end{verbatim}
with
\begin{verbatim}
\rule{4in}{3in}
\end{verbatim}
Alternatively, the \sty{mwe} package comes with some sample images
that you can use instead. For example, you could replace
\begin{verbatim}
\includegraphics{myImage}
\end{verbatim}
with
\begin{verbatim}
\includegraphics[height=3in]{example-image}
\end{verbatim}
(There are other test images provided by that package. See the
\sty{mwe} documentation for further details.)

If the file is a Bib\TeX\ file, then make a copy of the file, and
remove the entries one by one until you are left with the entry that
causes the problem. If the file is a CSV file, make a copy of the
file, and remove the rows one by one until you are left with the
problem row (but keep the header row if there is one.) You can then
send this abridged file with the minimal example or you can embed it
in the minimal example file using the \env{filecontents} or
\env{filecontents*} environment\footnote{The starred form doesn't
write extra comments in the file}. This environment takes one
argument which must be the name of the file. For example:
\begin{verbatim}
\documentclass{article}

\begin{filecontents*}{test.bib}
@article{sample,
 author={Ann Other},
 title={Sample Title},
 journal={Journal of Something},
 year=2014
}
\end{filecontents*}

\begin{document}
\cite{sample}

\bibliographystyle{plain}
\bibliography{test}
\end{document}
\end{verbatim}

Alternatively, if the problem is with the way the bibliography is
being formatted in general, you can use the \texttt{xampl.bib} file,
which should be available in your \TeX\ distribution.
\begin{verbatim}
\documentclass{article}

\begin{document}
\nocite{*}

\bibliographystyle{plain}
\bibliography{xampl}
\end{document}
\end{verbatim}

The \sty{glossaries} package also comes with some sample files that
you can use in a similar manner. These should be installed along
with the \sty{glossaries} package. For example:
\begin{verbatim}
\documentclass{article}

\usepackage{glossaries}
\makeglossaries

\loadglsentries{example-glossaries-brief}

\begin{document}
\glsaddall
\printglossaries
\end{document}
\end{verbatim}

\setnode{dummytext}
\section{Dummy Text}
\label{sec:dummytext}

Sometimes a problem may only occur at a certain place or after a
certain point, in which case you may need to create some dummy text
to pad out your example. If so, the \sty{lipsum} package is a useful tool.
This provides the command \cs{lipsum} which has an optional argument
that specifies the paragraph or the range of paragraphs to typeset. 

For example, suppose you are using the \cls{book} class and you don't
understand why the page number appears on the bottom of the first
page of the chapter and at the top of the second page. Then you
could illustrate this as follows: 
\begin{verbatim}
\documentclass{book}

\usepackage{lipsum}

\begin{document}
\chapter{Sample}

\lipsum[1-4]
\end{document}
\end{verbatim}
This will produce enough text to generate two pages. 

There is another dummy text package called \sty{blindtext} that
provides the commands \cs{blindtext} (for short blocks of text) and 
\cs{Blindtext} (for longer blocks of text). For example:
\begin{verbatim}
\documentclass{book}

\usepackage{blindtext}

\begin{document}
\chapter{Sample}

\Blindtext
\end{document}
\end{verbatim}
The \sty{blindtext} package also provides other commands to provide
a random document, dummy lists etc. See the \sty{blindtext}
documentation for further details.

There are other dummy text generators listed on
\htmladdnormallink{CTAN Topic
dummy-gen}{http://ctan.org/topic/dummy-gen}.

\setnode{documentation}
\section{Where Do I Find Package Documentation?}
\label{sec:documentation}

These days most package documentation is provided as a PDF file and,
if it is installed on your system, it can usually be obtained using
the \htmladdnormallink{texdoc
application}{http://www.dickimaw-books.com/latex/novices/html/texdoc.html}. 
If you have a \htmladdnormallink{terminal or command
prompt}{http://www.dickimaw-books.com/latex/novices/html/terminal.html},
you can access it by typing \texttt{texdoc} followed by the name of the
package. For example, to obtain the documentation for the
\sty{datetime} package run: 
\begin{verbatim}
texdoc datetime
\end{verbatim}
Sometimes this may produce just the documented code rather than the
user manual. For example:
\begin{verbatim}
texdoc flowfram
\end{verbatim}
will display the documented code. However, in this instance, the
first paragraph of that document tells you that the user manual is
in \texttt{ffuserguide.pdf} in which case
\begin{verbatim}
texdoc ffuserguide
\end{verbatim}
will produce the user manual.

In some cases (especially for older packages) the documentation may
be contained in a \texttt{README} file in the documentation directory or it
may be embedded as comments either at the start or the end of the
\texttt{.sty} or \texttt{.cls} file. 

Alternatively, if the documentation was not installed on your system,
you can obtain it from \htmladdnormallink{CTAN}{http://ctan.org/}.
You can either use the search box on the CTAN home page or you can use the URL
\texttt{http://ctan.org/pkg/}\meta{name} where \meta{name} is the name of the package.
For example, to obtain information on the \sty{glossaries} package, you
can use the URL \url{http://ctan.org/pkg/glossaries} and it will
provide links to the documentation for that package. 

There is also the \htmladdnormallink{TeXdoc Online
Resource}{http://texdoc.net/} which is a dedicated server for
querying and browsing \TeX\ and \LaTeX\ package information and
documentation, where you can search for packages or topics.

\setnode{errormessages}
\section{Understanding Error Messages}
\label{sec:errormessages}

\TeX\ and \LaTeX\ error messages can be cryptic, but sometimes it's
possible to at least find out where things have gone wrong by
studying the message. 

Consider the following document: 
\begin{verbatim}
\documentclass{article}

\newcommand{\example}[1]{#1}

\begin{document}
This is a sample document that contains a long 
command \example{with an error.

This is the next paragraph
\end{document}
\end{verbatim}
This produces the following error message: 
\begin{verbatim}
Runaway argument?
{with an error. \par This is the next paragraph \end {document}
! File ended while scanning use of \example.
<inserted text>
                \par
\end{verbatim}
The first line (\qt{Runaway argument?}) indicates the type of error.
A runaway argument is usually caused by a missing closing brace. The
next line indicates where \TeX\ got up to before things started to
go wrong. In this error message there is no line number but you can
use the information that has been supplied to help you track where
the error might be. Copy the first part of this line (say
\verb|{with an error|) and paste it into your editor's search
function. This should take you to the relevant line where you can
see that there is no closing brace.

Suppose, instead, the document looked like: 
\begin{verbatim}
\documentclass{article}

\newcommand*{\example}[1]{#1}

\begin{document}
This is a sample document that contains a short 
command \example{with an error.

This is the next paragraph
\end{document}
\end{verbatim}
In this case the error message is:
\begin{verbatim}
Runaway argument?
{with an error.
! Paragraph ended before \example was complete.
<to be read again>
                   \par
l.8
\end{verbatim}
In this example, the error message includes the line number where
things started to go wrong (l.8) so I can use my text editor's \qt{go
to line} function. 

Sometimes the line number given in the error message doesn't
correspond to the line number where the error actually occurs. For
example, consider the following document: 
\begin{verbatim}
\documentclass{report}

\author{A.N. Author}
\title{A sample document with a \badcommand}
\date{14th November, 2008}

\begin{document}
\maketitle
\end{document}
\end{verbatim}
In this document the error is an undefined command (\cs{badcommand})
occurring on line~4. However, the error message is: 
\begin{verbatim}
! Undefined control sequence.
\@title ->A sample document with a \badcommand

l.8 \maketitle
\end{verbatim}
which indicates that the problem occurs on line~8. This is because
\TeX\ doesn't actually try to interpret \cs{badcommand} until line~8 when
\cs{maketitle} tries to typeset the title page. 

When this type of situation occurs, it may be necessary to do a
little bit of detective work to try to trace the problem. In the
above example, there are two methods to try: 
\begin{enumerate}
\item The first line of the error message states the nature of the
error (an undefined control sequence) and the second line indicates
that the undefined control sequence is \cs{badcommand}. You can then use
your text editor to search for any instances of \cs{badcommand} and
replace it with the correct command. Alternatively, if you have
forgotten to use a package that defines the command or, in the case
of a custom command, you have forgotten to define the command, then
do so. 

\item The last line of the error message states that the problem was
encountered on line~8 which contains the command \cs{maketitle}. What
commands affect \cs{maketitle}? For the standard classes, such as
report, these are: \cs{author}, \cs{title} and \cs{date}, so inspect the code
where these commands are used. Try commenting out all but one of the
commands and see if the error still occurs. For example, if I
comment out the lines containing the title and date, the error goes
away, but if I comment out the author and date instead, the error
remains. This means that the error is in the title. 
\end{enumerate}

If you have many files that have been input (using, for example,
\cs{input} or \cs{include}) you may find it a little difficult to
work out which file the line number is referencing. Does l.8 refer
to line~8 of my main file \texttt{mydoc.tex} or does it refer to
line~8 of one of the files that have been input? In this case, you
may find it easier to use the \texttt{-file-line-error} option when
you run \LaTeX\ (or PDF\LaTeX\ etc):
\begin{verbatim}
pdflatex -file-line-error mydoc.tex
\end{verbatim}
Now the error will be preceded by the file name and line number in
the form:
\begin{verbatim}
./mydoc.tex:8: Undefined control sequence.
\end{verbatim}

For further information on understanding error messages, see
\htmladdnormallink{How
to approach
errors}{http://www.tex.ac.uk/cgi-bin/texfaq2html?label=erroradvice} on the UK TeX FAQ. There is also a list of some
\htmladdnormallink{common error
messages}{http://www.dickimaw-books.com/latex/novices/html/commonerrors.html}
in the document \htmladdnormallink{LaTeX for Complete
Novices}{http://www.dickimaw-books.com/latex/novices/}
which is available from the same site as this document. 

\setnode{license}
\input{fdl}

\end{document}
